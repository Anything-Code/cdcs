\documentclass[conference]{IEEEtran}
\IEEEoverridecommandlockouts
% The preceding line is only needed to identify funding in the first footnote. If that is unneeded, please comment it out.
\usepackage{cite}
\usepackage{hyperref}
\usepackage{amsmath,amssymb,amsfonts}
\usepackage{algorithmic}
\usepackage{graphicx}
\usepackage{textcomp}
\usepackage{xcolor}

\hypersetup{
  colorlinks   = true, %Colours links instead of ugly boxes
  urlcolor     = blue, %Colour for external hyperlinks
  linkcolor    = red, %Colour of internal links
  citecolor   = red %Colour of citations
}

\def\BibTeX{{\rm B\kern-.05em{\sc i\kern-.025em b}\kern-.08em
    T\kern-.1667em\lower.7ex\hbox{E}\kern-.125emX}}
\begin{document}

\title{Quality Assurance for Spatial Research Data\\
{\footnotesize A summary - Original by: M. Wagner and C. Henzen - Published in: ISPRS International Journal of Geo-Information, vol. 11, no. 6, p. 334, 2022 \cite{sjr}}
\thanks{DOI of \cite{wagner2022quality}: \href{https://doi.org/10.3390/ijgi11060334}{10.3390/ijgi11060334} - \href{https://github.com/Anything-Code/quality-assurance-for-spatial-research-data-a-summary/blob/master/summary.tex}{LaTeX code}}
}

\author{\IEEEauthorblockN{1\textsuperscript{st} Niklas Lübcke}
\IEEEauthorblockA{\textit{Faculty of Information, Media \& Design} \\
\textit{SRH Heidelberg}\\
Heidelberg, Germany \\
niklas.lubcke@gmail.com}
\and
\IEEEauthorblockN{2\textsuperscript{nd} Michael Wagner}
\IEEEauthorblockA{\textit{Center for Information Services and High Performance Computing (ZIH)} \\
\textit{Technische Universität Dresden}\\
Dresden, Germany \\
michael.wagner@tu-dresden.de}
\and
\IEEEauthorblockN{3\textsuperscript{rd} Christin Henzen}
\IEEEauthorblockA{\textit{Working Group Geo UX, German UPA} \\
\textit{Technische Universität Dresden}\\
Dresden, Germany \\
christin.henzen@tu-dresden.de}
}

\maketitle

\begin{abstract}
    The authors of \cite{wagner2022quality} propose an Assertion Framework (\ref{AF}) and a Quality Assurance Workflow (\ref{QAW}) for spatial data sources in Earth System Science (ESS). The framework is comprised of an evaluation of the 5-star rating system for open data by Tim Berners-Lee, a revised maturity matrix including FAIR (Findable, Accessible, Interoperable and Reusable) criteria and a spatial data quality matrix where maturity levels are related to quality metrics. The metrics of openness, maturity and quality are then mapped to phases within the research data lifecycle to produce the proposed QA workflow, which the authors have implemented in the interactive questionnaire tool RDMO (Research Data Management Organiser).
\end{abstract}

\begin{IEEEkeywords}
quality assurance; data maturity; maturity matrix; spatial data quality; FAIR
\end{IEEEkeywords}

\section{Introduction}
M. Wagner and C. Henzen recognised the level of quality of spatial data as an indication of the relevance of the extracted results of scientific work in the field. They suggest that there is a vacuum for tools to assess the quality of spatial data sources. In addition, despite the high quality of scientific data creation and manipulation workflows, the authors identified the monitoring and reporting of adequate information as a challenge in research data management (RDM). This leads to information being output only at the end or after the end of the data lifecycle.

Both of the challenges are addressed in \cite{wagner2022quality}. A framework for quantifying the quality of data sources is proposed, where sources can be rated from 1 to 5 stars. A QA workflow that includes monitoring and reporting during the data lifecycle is also presented.

\section{Results}

\subsection{Assertion Framework} \label{AF}

The IEEEtran class file is used to format your paper and style the text. All margins, 
column widths, line spaces, and text fonts are prescribed; please do not 
alter them. You may note peculiarities. For example, the head margin
measures proportionately more than is customary. 

Before you begin to format your paper, first write and save the content as a 
separate text file. Complete all content and organizational editing before 
formatting. Please note sections \ref{AF}-\ref{QAW} below for more information on 
proofreading, spelling and grammar.

\subsection{Quality Assurance Workflow}\label{QAW}
Define abbreviations and acronyms the first time they are used in the text, 
even after they have been defined in the abstract. Abbreviations such as 
IEEE, SI, MKS, CGS, ac, dc, and rms do not have to be defined. Do not use 
abbreviations in the title or heads unless they are unavoidable \cite{wagner2022quality}.

\begin{itemize}
\item Use either SI (MKS) or CGS as primary units. (SI units are encouraged.) English units may be used as secondary units (in parentheses). An exception would be the use of English units as identifiers in trade, such as ``3.5-inch disk drive'' \cite{sjr}.
\end{itemize}

\subsection{Proof Of Concept}\label{POC}

Complete all content and organizational editing before 
formatting. Please note sections \ref{AF}-\ref{QAW} below for more information on 
proofreading, spelling and grammar.

\bibliographystyle{IEEEtran}
\bibliography{references}

\renewcommand{\refname}{Appendix}

\begin{thebibliography}{00}
\makeatletter
\addtocounter{\@listctr}{2}
\makeatother
\bibitem{b1} Infographic of the proposed Assertion Framework and the QA Workflow
\end{thebibliography}

\end{document}
