\documentclass[conference]{IEEEtran}
\IEEEoverridecommandlockouts
% The preceding line is only needed to identify funding in the first footnote. If that is unneeded, please comment it out.
\usepackage{cite}
\usepackage{hyperref}
\usepackage{amsmath,amssymb,amsfonts}
\usepackage{algorithmic}
\usepackage{graphicx}
\usepackage{textcomp}
\usepackage{xcolor}

\hypersetup{
  colorlinks   = true, %Colours links instead of ugly boxes
  urlcolor     = blue, %Colour for external hyperlinks
  linkcolor    = red, %Colour of internal links
  citecolor   = red %Colour of citations
}

\def\BibTeX{{\rm B\kern-.05em{\sc i\kern-.025em b}\kern-.08em
    T\kern-.1667em\lower.7ex\hbox{E}\kern-.125emX}}
\begin{document}

\title{Quality Assurance for Spatial Research Data\\
{\footnotesize A summary - Original by: M. Wagner and C. Henzen - Published in: ISPRS International Journal of Geo-Information, vol. 11, no. 6, p. 334, 2022 \cite{sjr}}
\thanks{DOI of \cite{wagner2022quality}: \href{https://doi.org/10.3390/ijgi11060334}{10.3390/ijgi11060334} - LaTeX code of this summary: \href{https://github.com/Anything-Code/quality-assurance-for-spatial-research-data-a-summary/blob/master/summary.tex}{GitHub}}
}

\author{\IEEEauthorblockN{1\textsuperscript{st} Niklas Lübcke}
\IEEEauthorblockA{\textit{Faculty of Information, Media \& Design} \\
\textit{SRH Heidelberg}\\
Heidelberg, Germany \\
niklas.lubcke@gmail.com}
\and
\IEEEauthorblockN{2\textsuperscript{nd} Michael Wagner}
\IEEEauthorblockA{\textit{Center for Information Services and High Performance Computing (ZIH)} \\
\textit{Technische Universität Dresden}\\
Dresden, Germany \\
michael.wagner@tu-dresden.de}
\and
\IEEEauthorblockN{3\textsuperscript{rd} Christin Henzen}
\IEEEauthorblockA{\textit{Working Group Geo UX, German UPA} \\
\textit{Technische Universität Dresden}\\
Dresden, Germany \\
christin.henzen@tu-dresden.de}
}

\maketitle

\begin{abstract}
    The authors of \cite{wagner2022quality} propose an Assertion Framework (\ref{AF}) and a Quality Assurance Workflow (\ref{QAW}) for spatial data sources in Earth System Science (ESS). The framework is comprised of an evaluation of the 5-star rating system for open data by Tim Berners-Lee, a revised maturity matrix including FAIR (Findable, Accessible, Interoperable and Reusable) criteria and a spatial data quality matrix where maturity levels are related to quality metrics. The metrics of openness, maturity and quality are then mapped to phases within the research data lifecycle to produce the proposed QA workflow.
\end{abstract}

\begin{IEEEkeywords}
quality assurance; data maturity; maturity matrix; spatial data quality; FAIR
\end{IEEEkeywords}

\section{Introduction}
M. Wagner and C. Henzen recognised the level of quality of spatial data as an indication of the relevance of the extracted results of scientific work in the field. They suggest that there is a vacuum for tools to assess the quality of spatial data sources. In addition, despite the high quality of scientific data creation and manipulation workflows, the authors identified the monitoring and reporting of adequate information as a challenge in research data management (RDM). This leads to information being output only at the end or after the completion of the data lifecycle.

\section{Results}

The primary results of \cite{wagner2022quality} are the following, with the mapping of openness metrics to maturity and quality metrics being summarised in the Assertion Framework (\ref{AF})

\begin{enumerate}
    \item Assertion Framework (\ref{AF})
    \item Quality Assurance Workflow (\ref{QAW})
    \item Proof of Concept (\ref{POC})
\end{enumerate}

\subsection{Assertion Framework} \label{AF}

The aim of the Assertion Framework is to provide a tool for assessing the quality of spatial data sources according to the following categories:

\begin{enumerate}
    \item Openness
    \item Data maturity
    \item Data quality
\end{enumerate}

The authors of \cite{wagner2022quality} do not calculate an average of all the metrics or categories included. Instead, the output of the framework is 3 independent ratings between 1 and 5 stars. The ratings are created by following a ladder principle, where a higher step adds another aspect to the metrics of the data source.

It is possible for categories to have limited rating ranges. As shown in \cite{wagner2022quality}, e.g. the openness category can only produce a rating between 2 and 4 stars.

\subsection{Quality Assurance Workflow}\label{QAW}

To create the best possible data sources as part of the scientific data lifecycle \cite{wagner2022quality}, the metrics of \ref{AF} are mapped to each phase of the lifecycle and its appropriate roles to create a workflow that acts as a QA for scientific teams.

Furthermore, the workflow was implemented as a plugin in the interactive questionnaire tool RDMO (Research Data Management Organiser), a Python software for self-hosting on the World Wide Web.

\subsection{Proof Of Concept}\label{POC}

To apply their proposed framework, the authors rate and discuss SPAM2010 (A land use dataset) 3.5 stars in the openness category, 4 to 5 stars in the maturity category and 5 stars in the quality category.

\bibliographystyle{IEEEtran}
\bibliography{references}

\renewcommand{\refname}{Appendix}

\begin{thebibliography}{00}
\makeatletter
\addtocounter{\@listctr}{2}
\makeatother
\bibitem{figure} Figure of the proposed Assertion Framework and the QA Workflow
\end{thebibliography}

\end{document}
