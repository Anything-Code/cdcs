\documentclass[conference]{IEEEtran}
\IEEEoverridecommandlockouts
% The preceding line is only needed to identify funding in the first footnote. If that is unneeded, please comment it out.
\usepackage{cite}
\usepackage{amsmath,amssymb,amsfonts}
\usepackage{algorithmic}
\usepackage{graphicx}
\usepackage{textcomp}
\usepackage{xcolor}
\def\BibTeX{{\rm B\kern-.05em{\sc i\kern-.025em b}\kern-.08em
    T\kern-.1667em\lower.7ex\hbox{E}\kern-.125emX}}
\begin{document}

\title{Quality Assurance for Spatial Research Data\\
{\footnotesize M. Wagner and C. Henzen, “Quality assurance for spatial research data,”
ISPRS International Journal of Geo-Information, vol. 11, no. 6, p. 334,
2022}
\thanks{Identify applicable funding agency here. If none, delete this.}
}

\author{\IEEEauthorblockN{1\textsuperscript{st} Niklas Lübcke}
\IEEEauthorblockA{\textit{Faculty of Information, Media \& Design} \\
\textit{SRH Heidelberg}\\
Heidelberg, Germany \\
niklas.lubcke@gmail.com}
\and
\IEEEauthorblockN{2\textsuperscript{nd} Michael Wagner}
\IEEEauthorblockA{\textit{Center for Information Services and High Performance Computing (ZIH)} \\
\textit{Technische Universität Dresden}\\
Dresden, Germany \\
michael.wagner@tu-dresden.de}
\and
\IEEEauthorblockN{3\textsuperscript{rd} Christin Henzen}
\IEEEauthorblockA{\textit{Working Group Geo UX, German UPA} \\
\textit{Technische Universität Dresden}\\
Dresden, Germany \\
christin.henzen@tu-dresden.de}
}

\maketitle

\begin{abstract}
    For spatial data sources in Earth System Science (ESS), the authors (M. Wagner and C. Henzen) propose a revised maturity matrix including FAIR (Findable, Accessible, Interoperable and Reusable) criteria and a spatial data quality matrix where maturity levels are related to quality metrics. Both metrics are then mapped to phases within the research data lifecycle to produce a QA workflow, which the authors have implemented in the interactive questionnaire tool RDMO (Research Data Management Organiser). data lifecycle
\end{abstract}

\begin{IEEEkeywords}
quality assurance; data maturity; maturity matrix; spatial data quality; FAIR
\end{IEEEkeywords}

\section{Introduction}
This document is a model and instructions for \LaTeX.
Please observe the conference page limits. 

\section{Ease of Use}

\subsection{Maintaining the Integrity of the Specifications}

The IEEEtran class file is used to format your paper and style the text. All margins, 
column widths, line spaces, and text fonts are prescribed; please do not 
alter them. You may note peculiarities. For example, the head margin
measures proportionately more than is customary. This measurement 
and others are deliberate, using specifications that anticipate your paper 
as one part of the entire proceedings, and not as an independent document. 
Please do not revise any of the current designations.

\section{Prepare Your Paper Before Styling}
Before you begin to format your paper, first write and save the content as a 
separate text file. Complete all content and organizational editing before 
formatting. Please note sections \ref{AA}--\ref{SCM} below for more information on 
proofreading, spelling and grammar.

Keep your text and graphic files separate until after the text has been 
formatted and styled. Do not number text heads---{\LaTeX} will do that 
for you.

\subsection{Abbreviations and Acronyms}\label{AA}
Define abbreviations and acronyms the first time they are used in the text, 
even after they have been defined in the abstract. Abbreviations such as 
IEEE, SI, MKS, CGS, ac, dc, and rms do not have to be defined. Do not use 
abbreviations in the title or heads unless they are unavoidable \cite{wagner2022quality}.

\subsection{Units}
\begin{itemize}
\item Use either SI (MKS) or CGS as primary units. (SI units are encouraged.) English units may be used as secondary units (in parentheses). An exception would be the use of English units as identifiers in trade, such as ``3.5-inch disk drive'' \cite{sjr}.
\item Avoid combining SI and CGS units, such as current in amperes and magnetic field in oersteds. This often leads to confusion because equations do not balance dimensionally. If you must use mixed units, clearly state the units for each quantity that you use in an equation.
\item Do not mix complete spellings and abbreviations of units: ``Wb/m\textsuperscript{2}'' or ``webers per square meter'', not ``webers/m\textsuperscript{2}''. Spell out units when they appear in text: ``. . . a few henries'', not ``. . . a few H''.
\end{itemize}

\bibliographystyle{IEEEtran}
\bibliography{references}

\end{document}
